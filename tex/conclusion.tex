%
% Conclusions
%

\chapter{Conclusion}
%One of the major problems with the standard model of particle physics is the origin of particle mass. The Higgs mechanism was designed to solve these but has aesthetic problems. Supersymmetry, and hence the MSSM, was created in part to help solve these issues. The LHC and the CMS detector are being constructed, in part, to verify, or refute, these theories.

%It has been seen that CMS will have an unprecedented data rate that requires a large, scalable computing platform. The grid paradigm has been adopted for this and the LCG/EGGE grids developed. 

%CMS had developed a plan for how to make best use of the available resources. This calls for easy access to these resources for non-expert users. 

One of the most important CMS grid activities is that of distributed user analysis. GROSS was created as a prototype distributed analysis tool. This tool performed well, submitting almost 4,000 jobs with a success rate of almost 90\%. Nevertheless it was decided to discontinue development in favour of a similar tool. 

The experience gained from GROSS development and usage was used in the redesign of one of the core pieces of CMS software, BOSS. This software was designed to be used within the CMS production system and was adopted by GROSS for use with analysis. BOSS is currently used by both production and analysis systems within CMS and runs tens of thousands of jobs per week.

%The MSSM introduces five Higgs bosons two of which, the $A$ and $H$, in certain scenarios couple strongly to down type fermions. This provided the motivation to study $\tau$'s and to ensure that CMS could make an accurate measurement of their energies. 

%This provided the motivation for studying the decay $A/H \rightarrow \tau\tau \rightarrow$ two jets.

The $\tau$ energy scale has been investigated and Monte Carlo corrections produced for use by the CMS physics community. It is envisaged that this correction will be maintained until CMS particle flow algorithms become sufficiently sensitive. During experiment startup when the particle flow is still being investigated it may be possible to use QCD jet + photon events along with $\tau$ tagging to calibrate the energy scale. Investigation has shown that this may be a possibility but that the $\tau$ tagging criteria will have to be improved.

The study in this thesis has shown that both the $A$ and $H$, if they exist, may be seen at CMS in the channel $gg \rightarrow A/H \rightarrow \tau\tau \rightarrow$ two jets with 60\fb of data. This study has been performed using an \MET selection instead of, as studied previously, b-tagging. It has been seen that this channel provides a 5$\sigma$ discovery significance for $m_A$ = 200\GeVcc with $\tan{\beta} \ge 22$, $m_A$ = 500\GeVcc with $\tan{\beta} \ge 31$ and $m_A$ = 800\GeVcc with $\tan{\beta} \ge 48$. This is better than the previous study except at $m_A$ = 200\GeVcc. 

It is envisaged that, once running, CMS will implement an energy flow algorithm which will improve its \MET scale and resolution. This should improve the performance of this study at low $m_A$. This study will be combined with the b-tagging and other studies to obtain the greatest reach in the \plane plane.