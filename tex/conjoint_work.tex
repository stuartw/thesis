						%Conjoint work statement
		
Throughout this thesis where the work of others is presented references to the originating work are shown. Where my own work has been published references are also listed for completeness. The areas of this thesis that are my own work are given below.

 %This particularly applies to the $\tau$ study which illustrate the optimum reconstruction cone, the simulated $\tau$ energy calibration and the proposed method for obtaining the calibration from real data. The sections of this thesis that are my own work are given below.
						
I was responsible for the input data handling (Section 4.5.2), submitted workflow (Section 4.5.5) and local farm submission (4.5.7) parts of the distributed analysis package (GROSS~\cite{CHEP04_TALLINI}). Shortly after the first public release the main developer left leaving the author to take up this role. The project was then developed further, in response to feedback from users. All further work (Sections 4.6 onwards), including major structural and user interface changes necessary due to the new PubDB system, were implemented by me except for the low level http PubDB interaction which was written by a new developer under my guidance. 
% and inline with the evolution of the CMS computing model.

When development was moved to the BOSS~\cite{citeulike:880984} project, I was heavily involved in the design and architecture decisions (Section 5.3). The task concept and the use of XML for the task description was my idea. I was then responsible for writing the XML processing implementation (Section 5.4) as well as the multi-language API (Section 5.5).

In the $\tau$ study, Chapter 7, I determined the optimum reconstruction cone (Section 7.1), the simulated $\tau$ energy scale and a Monte Carlo calibration for the energy scale (Section 7.3.1). I also evaluated the effectiveness of a proposed method for obtaining the calibration from real data (Section 7.3.2).

The Higgs study (Section 8) involved a comparison with a previous study using a different experimental signature. As such this study is all my own work which further develops previous studies.