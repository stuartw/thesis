%
% Introduction
%

%\chapter{Thesis layout}
The Standard Model has been tested to great precision and for the most part agrees well with experiment. To provide particles with mass it relies on the Higgs mechanism~\cite{citeulike:918519}. This introduces a new particle, the Higgs boson. However, there are features of the Higgs mechanism which perturb some physicists and so a complementary theory, Supersymmetry~\cite{citeulike:681336}, has been proposed. This theory requires additional Higgs bosons. The nature of both these theories and experimental results suggest that the Higgs boson(s) have masses below 1\,\TeVcc.

The Large Hadron Collider~\cite{LHC_CONCEPT} (LHC) currently being built at CERN has been designed to test these theories. The Compact Muon Solenoid~\cite{CMS_LOI,CMS_TP} (CMS) detector is located at the LHC and should either discover, or rule out, both the Higgs mechanism and low energy Supersymmetry. A description of the design and performance of this detector, concentrating on systems relevant to the analysis presented later, is given in Chapter 1.

The processes that will identify new physics are rare and will suffer from large backgrounds of well-understood physics. In order to obtain sufficient signal events the LHC must operate at unprecedented energy, luminosity and event rate. CMS will produce data at a rate of 1--10\,PB a year for many years. This requires a new paradigm in data storage and analysis, which is provided by a worldwide grid of computational and storage resources made available to any physicist for their research. The grid technology and software designed for the LHC is described in Chapter 2.

The way in which CMS will use the grid is described in Chapter 3. CMS required an application to allow its data analysis software to take advantage of distributed resources. A prototype distributed analysis tool, called GROSS, was developed and is described in Chapter 4. 
%During this time the project was further developed, influenced by the results of a community evaluation by CMS physicists and the evolution of the CMS computing model.

Many of the features demonstrated in GROSS were implemented in another CMS software system called BOSS. This process is described in Chapter 5. 
%The author was heavily involved in the design and architecture as well as the implementations of the XML processing and multi-language API.

Chapter 6 describes the Higgs mechanism, its problems and how supersymmetry may solve them. There are many flavours of supersymmetry, the simplest of which is known as the Minimal Supersymetric Standard Model (MSSM). The description of the MSSM in this chapter concentrates on areas useful for a search for heavy neutral MSSM Higgs bosons, the $A$ and $H$. It is shown why the $\tau$ lepton will provide a useful signature of these particles. Chapter 7 describes properties of the $\tau$, how to identify them at CMS and how to measure their energies correctly.
% The author was responsible for determining the optimum reconstruction cone, the Monte Carlo $\tau$ energy calibration and the evaluation of a proposed method to obtain the calibration from real data.

A study of CMS' ability to find these particles is presented in Chapter 8. This study investigates the possibility of seeing the process $gg \rightarrow A/H \rightarrow \tau \tau \rightarrow$ 2 jets at CMS. It builds on a previous study which investigated $gg \rightarrow bbA/H \rightarrow \tau \tau \rightarrow$ 2 jets with b-tagging used for event selection. The study presented here investigated the feasibility of replacing the b-tagging selection with one based on missing energy. This has the advantage of including the gluon fusion, $gg \rightarrow A/H$, production mechanism. The MSSM parameter space that CMS will be able to search in the early years of data-taking with this strategy is also presented. 

%This study was guided by the CMS Higgs group convenor who was also the author of a similar previous study (with a different selection strategy). As such the study was the authors work but guided and inspired by the previous author.
This layout has been chosen to provide a more flowing thesis. The theory chapter was placed just before the $\tau$ study and analysis chapters so that it was close to the place where the information contained within was put to use.